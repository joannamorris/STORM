\documentclass[]{article}

\usepackage[letterpaper, portrait, margin=.75in]{geometry}
\usepackage{parskip}
\usepackage[style=authoryear, backend=biber]{biblatex}
\addbibresource{metaphor.bib}
\usepackage{phonetic}

%opening
\title{That Project was a Roller coaster: An ERP test of Deliberate Metaphor Theory}
\author{Sophie Greene, Joanna Morris, Daniel Altshuler}

\begin{document}
	

\maketitle

\begin{abstract}
%\setlength{\parskip}{1\baselineskip}	

	Deliberate Metaphor Theory (DMT), holds that deliberate metaphors are those that are intended in their production to be explicit metaphors, and that are understood and processed as such by the comprehender. Deliberate metaphors thus serve as a distinct rhetorical device with an explicit communicative goal. DMT posits that it is only for these deliberate metaphors that cross-domain mappings are recruited, as a result of the speaker signaling to the hearer\textemdash via distinct pragmatic cues\textemdash that the source domain should be explicitly represented and mapped to the target. In this ERP study we cued deliberate metaphors by using the word ‘like’ in sentences of the form ‘a [target domain] is like a [source domain]’, and compared N400 responses to words serving as the source domain in both deliberate and in non-deliberate metaphors. Although amplitudes of the N400 ERP component were more negative for metaphors than for literal sentences, the amplitude of N400 responses to deliberate and non-deliberate metaphors did not differ. These data suggest that contrary to the claims of DMT, deliberate and non-deliberate metaphors may be processed in similar ways.

\end{abstract}

\section{Introduction}

	
    A metaphor is a novel linguistic expression used outside of its normal conventional meaning to refer to an entity, property or relation it does not literally denote. Consider this except from Shakespeare's Romeo and Juliet: ``It is the east, and Juliet is the sun."  Shakespeare means to convey not that Juliet is a celestial body, but that to Romeo, her beauty is greater, or `outshines' that of all other entities.  
	Metaphors have generally been considered rare, ornamental linguistic phenomena, restricted to the domains of poetry and literature.  However, in conceptual metaphor theory (CMT), metaphor is recast as fundamentally conceptual in nature, in that it establishes correspondences between concepts from disparate domains of knowledge\parencite{lakoff_conceptual_1980}.  
	
	CMT claims that our entire conceptual system is metaphorical and that these conceptual metaphors are present in everyday, conventional communication. In this framework, familiar metaphors, e.g. ``we have come a long way" are linguistic manifestations of underlying conceptual relationships such as ``love is a journey" or ``argument is war." 
	
	This understanding of metaphor is at odds with the `standard pragmatic model' of metaphor in which conversations are considered to be cooperative efforts undertaken in pursuit of  a common communicative goal \parencite{grice_logic_1975}.  Participants in a conversation make contributions that advance the shared goal, by offering information that they believe to be true as well as relevant, informative and clear. If a listener believes that a speaker adheres to the cooperative principle, the speaker's use of a literally false nominal metaphor of the form `an a is a b', forces the listener to search for an alternative metaphorical interpretation that renders the utterance true. Under this model, a hearer first attempts to process a metaphor literally, but if the literal interpretation turns out to be false, an interpretation is sought that renders the utterance true \parencite{searle_metaphor_1993}. This alternative interpretation is attained by engaging in an implicit comparison via which the hearer compares the features of `a’, the tenor, with those of `b’, the vehicle \parencite{ortony_beyond_1979, gentner_structure-mapping:_1983}. Thus, metaphors are similes with the comparison word “like” omitted and metaphors are processed in the same way that their simile counterparts would be \parencite{glucksberg_how_1993} (Glucksberg and Keysar, 1993).
	
	Because of the extra steps and effort involved in processing a metaphorical sentence, metaphors should take longer to process than literal sentences. But studies conducted by Gibbs and colleagues (1982; 1984) comparing reading times for literal and nonliteral sentences (which included metaphor, irony, and indirect requests) found equivalent reading times for both types. These results suggest that accessing a literal interpretation is not obligatory when processing nonliteral sentences. 
	
	Moreoever, comparison theories are challenged by experimental data showing that “A is a B” metaphors are processed faster than similes, and that participants prefer metaphors in direct metaphor form over metaphors in simile form (Chiappe and Kennedy, 2000; Johnson, 1996).
	
	However, an alternative view of metaphor interpretation is that metaphors are not processed as comparisons, but rather as categorical assertions, where the entity ‘a’ is claimed to be a member of the superordinate category ‘b’.  
	
	Bowdle and Gentner (2005) unify these positions by asserting that as metaphors are conventionalized, there is a shift in mode of processing from comparison to categorization; novel metaphors are processed by comparison alone, whereas conventional metaphors can be processed by either mode.  However, whether a metaphor is processed by comparison or categorization is determined not only by its novelty, but also by its form.  When a conventional metaphor takes the form of a simile, it is processed by comparison; in contrast when a novel metaphor takes the form of a categorical assertion such as ‘an a is a b’, people process the target as an instance of the source category, i.e. by categorization.
	
	Steen (2008) argues that the process of metaphor interpretation, i.e. comparison vs. categorisation, is driven by both the metaphorical form that a speaker chooses to use, as well as the communicative function that the metaphor is intended to serve.
	The deliberate use of a metaphor is a rhetorical strategy that ‘changes the addressee’s perspective on the referent or topic that is the target of the metaphor, by making the addressee look at it from a different conceptual domain or space.” (Steen, 2008, p. 222)
	Hence, a deliberate metaphor is one that is used by speakers for the explicit purpose of being recognized and understood as metaphorical.
	Steen (2008) argues that analogies and similes, are the clearest examples of deliberate metaphors, because they explicitly invite comparisons between the target and source domains. 
	
	Taken together these findings lead to what Steen terms ‘the paradox of metaphor’, i.e. contrary to the claims of CMT, most words and phrases typically seen as conveying metaphorical meaning, by virtue of not being intentionally metaphorical, are not processed as cross-domain mappings and, therefore, are not truly ‘metaphorical’. 
	
	If so, we should find differences between (a) the processing of metaphors when preceded by pragmatic cues signaling the intention of the speaker for the utterance to be interpreted metaphorically, and (b) the processing of metaphors in the absence of such cues. 
	To test this hypothesis we examined event-related potential (ERP) responses to cued (deliberate) and uncued (non-deliberate) metaphors.
	
	ERPs are voltage changes in the on-going electroencephalogram that are time-locked to the onset of a sensory or motor event; they allow us to closely tie cognitive processes to brain function.
	
	Two components have been commonly reported for metaphors, the N400 and  the P600/LPC (Pynte et al.,1996; Coulson and Van Petten, 2002; De Grauwe et al., 2010; Schmidt-Snoek et al., 2015).
	Changes in the amplitude of the N400 may reflect the ease of mapping the meaning(s) of incoming words onto semantic memory structure and sentence and discourse level context, while the P600 has been traditionally linked to syntactic reanalysis, although recent studies have suggested that it also reflects semantic and interpretation processes, such as sentence-level interpretation conflicts (Frenzel et al., 2011) and integration in the wider discourse model and communicative context.
	Most studies reported a biphasic N400-P600 effect, assumed to link different stages in conceptual mapping (Coulson and Van Petten, 2002; De Grauwe et al., 2010). Other authors reported an N400 response only (Pynte et al., 1996), or a P600 only, described as a form of a reanalysis stage (Yang et al., 2013).
	
	In a recent study looking at the role of context in metaphor processing, Bambini et al. (2016) found a greater N400 for metaphors in a minimal versus a supportive context, suggesting that discourse cues may modulate metaphorical processing.
	
	Participants read 100 metaphorical and 100 literal sentences, with and without a pragmatic cue to metaphorical status.
	Sentences were presented one word at a time, at a rate that was meant to simulate natural reading, and participants were instructed to read for comprehension.
	
	The materials were 100 metaphor sentences, 100 non-metaphor sentences, and 200 filler sentences. Metaphor sentences were of the structure “A is a B,” and were presented either with or without the pragmatic marker “like.” 
	Non-metaphor sentences were of the same structure, and were also presented either with or without the pragmatic marker “like.”
	10\% of sentences were followed by comprehension questions to ensure the participant was paying attention. These were simple yes-or-no questions directly related to the preceding sentence.
	
	Each word was presented for 200 ms, and the interword interval was dependent on the length of the preceding word, at 100 ms plus an additional 37 ms for each character in the word (see Coulson and Petten, 2002; Lai and Curran, 2013; Lai et al., 2009, and others). After the final word and interword interval, participants saw a symbol on the screen (“ ( - - ) ”) for 1000 ms that indicated a blink break. For trials with questions, following the blink break there was a blank screen for 1600 ms, and then the question appeared. The question remained on the screen for 2000 ms or until the participant made their response with the keyboard. The question was followed by a variable intertrial interval of 800-1200 ms.
	
	Metaphors generally elicit greater N400 amplitudes than literal sentences and this is what we found.
	However, we found show no differences in N400 amplitude between cued (deliberate) and uncued (non-deliberate) metaphors. 
	All metaphors, deliberate or not, produced uniformly greater N400s than non-metaphor sentences. Therefore, the results of this study do not provide any evidence in support of Deliberate Metaphor Theory. 
	
\printbibliography

\end{document}
